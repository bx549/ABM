\documentclass{article}
\usepackage{parskip}
\usepackage{fullpage}
\usepackage{amsmath}
\usepackage{amssymb}
\usepackage{latexsym,paralist}
\usepackage{comment}
\usepackage{fancyvrb}
\usepackage{graphicx}
\usepackage{tikz}
\usetikzlibrary{arrows.meta}
\pagestyle{empty}

\includecomment{solution}
\excludecomment{solution}  % comment this out to create the solutions

\newcommand{\bs}{\noindent\textbf{Solution.}~}
\newcommand{\es}{\hfill $\square$}

\setcounter{secnumdepth}{-1}

\begin{document}
\subsection{IE 5545 Changing the Network Structure}
Adding links to a game of strategic substitutes can change the
equilibrium structure in unexpected ways.  Consider the ``Best-shot''
public goods game in which a player will receive a benefit of 1 if she
or any of her neighbors take action 1 ( but she prefers that one of
her neighbors take action 1). In this game the utility to player
$i$ for taking action 0 or 1 is
\begin{align*}
  u_i(1,S_{N_i}) &= 1 - c\\
  u_i(0,S_{N_i}) &= \begin{cases}
    1 \quad \text{if $a_j=1$ for some $j \in N_i$}\\
    0 \quad \text{if $a_j=0$ for all $j \in N_i$}
    \end{cases}
\end{align*}
where $N_i$ are the neighbors of player $i$, $S_{N_i}$ are the
strategies of the neighbors of player $i$, $a_j$ is the action taken
by player j, and $0 < c < 1$ is the cost for taking action 1.  The
following situation is a pure strategy Nash equilibrium for this
game. Now, add a link that connects the two centrally located nodes
and then change the players' behavior (i.e. choice of action) in a
sequential manner until a new equilibrium is achieved.

\vspace{.5in}
\begin{center}
\begin{tikzpicture}
  \tikzstyle{every circle node}=[draw,inner sep=0pt,minimum size=7mm]
  
  \draw (-3,0) node[circle] (A) {1};

  \draw (-3,2.8) node[circle] (C) {0};
  \draw (-5,1.5) node[circle] (D) {0};
  \draw (-4.8,-1.8) node[circle] (E) {0};
  \draw (-1.2,-1.8) node[circle] (F) {0};
  \draw (-1,1.5) node[circle] (G) {0};

  \draw (5,0) node[circle] (B) {1};

  \draw (5,2.8) node[circle] (H) {0};
  \draw (3,1.5) node[circle] (I) {0};
  \draw (3.2,-1.8) node[circle] (J) {0};
  \draw (6.8,-1.8) node[circle] (K) {0};
  \draw (7,1.5) node[circle] (L) {0};

  \path (A) edge (C);
  \path (A) edge (D);
  \path (A) edge (E);
  \path (A) edge (F);
  \path (A) edge (G);

  \path (B) edge (H);
  \path (B) edge (I);
  \path (B) edge (J);
  \path (B) edge (K);
  \path (B) edge (L);
 
\end{tikzpicture}
\end{center}

\end{document}
