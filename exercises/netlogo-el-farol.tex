\documentclass{article}
\input{/home/darin/MEGA/isye/preamble-homework.tex}

\includecomment{solution}
\excludecomment{solution}  % comment this out to create the solutions

\begin{document}
\subsection{Making decisions using inductive reasoning}

People don't always make perfect ``rational'' decisions. One theory is
that people recognize patterns or behaviors that lead to desired
outcomes and they apply those patterns to future decisions, even
though it may not be optimal to do so. This method of making decisions
is \emph{inductive}. For a good description, read the article
``\emph{Inductive Reasoning and Bounded Rationality}'' by Brian Arthur. It is
available on Canvas.  Don't be put off by the title; the article is an
easy read and it's only 6 pages long.  The purpose of this exercise is
to experiment with the theory in that article.

\begin{enumerate}
\item Look in the NetLogo Models Library under the Sample Models/Social Science folder.
  Open the ``El Farol'' model.
\item Read the information about the model.
\item Run the model with the default settings and observe the behavior of
  bar attendance.
\item Keep memory--size at its default value and vary
  number--strategies.  Set number--strategies to 2, and then set it to
  18. Notice the variation in bar attendance for these two settings and explain any
  difference. \label{numstrategies}
\item Now set number--strategies back to its default value of 10.  Set
  memory--size to 2 and then to 10. Explain any difference.
  \label{memsize}
\end{enumerate}

\begin{solution}
  \bs
  
  \ref{numstrategies}) With only two possible strategies for predicting
  attendance, there is little to no variation in bar attendance. If attendance
  is more predictable, then one strategy is likely to always perform better
  than the other, leading to more predictable behavior.

  \ref{memsize}) When memory--size is set to 2, there is large variation in attendance
  because the predictive strategies have little training data. When memory--size
  is set to 10, then variation in attendance is reduced; the predictive accuracy of the
  strategies is better estimated and patrons use the best strategies more consistently.
  
\end{solution}


\end{document}
